% !Mode:: "TeX:UTF-8"

\hitsetup{
  %******************************
  % 注意:
  %   1. 配置里面不要出现空行
  %   2. 不需要的配置信息可以删除
  %******************************
  %=========
  % 中文信息
  %=========
  ctitleone={非正弦电路下},%本科生封面使用
  ctitletwo={无功功率定义的发展沿革},%本科生封面使用
  ctitlecover={非正弦电路下无功功率定义的发展沿革},%放在封面中使用,自由断行
  ctitle={非正弦电路下无功功率定义的发展沿革},%放在原创性声明中使用
  csubtitle={电路课程论文(报告)}, %一般情况没有,可以注释掉
  cxueke={工学},
  csubject={电气工程及自动化},
  caffil={电气学院},
  cauthor={王浩瑞},
  csupervisor={霍炬教授},
  % 日期自动使用当前时间,若需指定按如下方式修改:
  %cdate={超新星纪元},
  cstudentid={1180610807},
  % 关键词用“英文逗号”分割
  ckeywords={非正弦电路,无功功率,科技史},
}

\begin{cabstract}
  在非正弦条件下,无功功率的定义无法直接套用正弦条件下无功功率定义的推广。为了明确非正弦条件下拥有具体物理意义的无功功率,学界进行了长达半个多世纪的研究。本文沿时间顺序系统梳理了非正弦电路无功功率定义的发展历史,对于理解非正弦电路理论发展有着启发作用。

\end{cabstract}
